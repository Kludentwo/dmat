\textbf{\Huge Problems}
%----------------------------------------------------------------------------------------
% Problem 1
%----------------------------------------------------------------------------------------
\section{Consider the following statements:}
$(1+2)^2-1^2=2^3$\\
$(1+2+3)^2-(1+2)^2=3^3$\\
$(1+2+3+4)^2-(1+2+3)^2=4^3$\\
\begin{enumerate}[(a)]
\item Based on the three statements given above, what is the next statement suggested by these?
\item What conjecture is suggested by these statements?
\item Verify the conjecture in (b).
\end{enumerate}
\textbf{\Large(a)}\\
We see that the next statement would be: $(1+2+3+4+5)^2-(1+2+3+4)^2=5^3$.\\
\textbf{\Large(b)}\\
The conjecture is $(1+2+3+4+5+...+n)^2-(1+2+3+4+...+n-1)^2=n^3$.\\
\textbf{\Large(c)}\\
Base case:\\
Let n = 2: $(1+2)^2-(1)^2 = 2^3$\\
$9-1 = 8$ so it is true.\\
We know from Theorem 6.3 that p(n) can be rewritten as
\begin{equation}
((n(n+1))/2)^2 - ((n-1(n))/2)^2 = n^3
\end{equation}
We assume p(k).
Let p(k + 1):\\
\begin{equation}
(k+1)^3 = (((k+1)(k+1+1))/2)^2 - (((k+1-1)(k+1+1-1))/2)^2 
\end{equation}
\begin{equation}
 = ((k^2+3k+2)/2)^2 - ((k^2+k)/2)^2 
\end{equation}
\begin{equation}
 = k^4/4 + 3k^3/2 + 13k^2/4 + 3k + 1 - (k^4/4 + k^3/2 + k^2/4)
\end{equation}
\begin{equation}
 = 2k^3/2 + 12k^2/4 + 3k + 1
\end{equation}
\begin{equation}
 = k^3 + 3k^2 + 3k + 1
\end{equation}
\begin{equation}
 = (k+1)^3
\end{equation}
Our conjecture was correct.\includegraphics[scale=0.70]{billeder/xzibit}

%----------------------------------------------------------------------------------------
% Problem 2
%----------------------------------------------------------------------------------------
\section{By an ordered partition of an integer $n\geq 2$ is meant a sequence of positive integers whose sum is n.}
For example, the ordered partitions of 3 are 3, 1 + 2, 2 + 1, 1 + 1 + 1.
\begin{enumerate}[(a)]
\item Determine the ordered partitions of 4.
\item Determine the ordered partitions of 5.
\item Make a conjecture concerning the number of ordered partitions of an integer $n\geq 2$
\end{enumerate}
\textbf{\Large(a)}\\
The ordered partitions of 4 are as following: \\
4,
 3 + 1, 1 + 3,
  2 + 2, 2 + 1 + 1, 1 + 2 + 1, 1 + 1 + 2,
   1 + 1 + 1 + 1\\
\textbf{\Large(b)}\\
The ordered partitions of 5 are as following: \\
5,
 4 + 1, 1 + 4,
  3 + 2, 3 + 1 + 1, 2 + 3, 1 + 1 + 3, 1 + 3 + 1
   2 + 2 + 1, 2 + 1 + 2, 1 + 2 + 2, 2 + 1 + 1 + 1, 1 + 2 + 1 + 1, 1 + 1 + 2 + 1, 1 + 1 + 1 + 2,
    1 + 1 + 1 + 1 + 1\\
\textbf{\Large(c)}\\
We see that 3 has 4 ordered partitions, 4 has 8 ordered partitions and 5 has 16 ordered partitions. This can also be written as $2^2$, $2^3$ and $2^4$. This leads to the conjecture that the number of ordered partitions of an integer $n\geq 2$ is $2^{n-1}$.\includegraphics[scale=0.70]{billeder/xzibit}


%----------------------------------------------------------------------------------------
% Problem 3
%----------------------------------------------------------------------------------------
\section{Express the following quantified statement in symbols:}
For every odd integer n, the integer 3n+1 is even.\\
$\forall n\in \mathbb{Z} : 2\nmid n , 2|3n+1$\\
\textbf{Part (b)} \\
Prove that the statement is true.
$3(2k+1)+1 = 6k +3 +1 = 2(3k+2)$.\includegraphics[scale=0.70]{billeder/xzibit}


%----------------------------------------------------------------------------------------
% Problem 4
%----------------------------------------------------------------------------------------
\section{Express the following quantified statement in symbols:}
There exists a positive integer n such that $3n+2^{n-2}$is odd.\\
$\exists n \in \mathbb{Z} : 2|n, 2\nmid 3n+2^{n-2}$\\
\textbf{Part (b)} \\
Prove that the statement is true.
$3*2k+2^{2k-2} = 6k+4^{k-1} = 6k+4^k*1/4$\\
$6k+1^k = 6k + 1 = 2(3k) + 1$.\includegraphics[scale=0.70]{billeder/xzibit}


%----------------------------------------------------------------------------------------
% Problem 5
%----------------------------------------------------------------------------------------
\section{Prove or disprove: The sum of every five consecutive integers is divisable by 5 and the sum of no six consecutive integers is divisable by 6}
We start with the sum of every five consectutive integers whether it is divisable by 5:\\
$(n)+(n+1)+(n+2)+(n+3)+(n+4) = 5n+10 = 5(n+2)$\\
So it is true. \includegraphics[scale=0.70]{billeder/xzibit}\\
We look at the sum of six consecutive integers whether it is divisable by 6:\\
$(n)+(n+1)+(n+2)+(n+3)+(n+4)+(n+5) = 6(n+2)+3$\\
This is false. It is disproven. \includegraphics[scale=0.70]{billeder/xzibit}\\


%----------------------------------------------------------------------------------------
% Problem 6
%----------------------------------------------------------------------------------------
\section{Consider the following statements:}
1 = 1,\\
1 + 3 = 4,\\
1 + 3 + 5 = 9,\\
1 + 3 + 5 + 7 = 16,\\
1 + 3 + 5 + 7 + 9 = 25.\\
\begin{enumerate}[(a)]
\item Based on the three statements given above, what is the next statement suggested by these?
1 + 3 + 5 + 7 + 9 + 11 = 36.
\item What conjecture is suggested by these statements?
$1 + 3 + 5 + 7 + 9 + ... + 2k + 1 = (k + 1)^2$. $\forall k \in \mathbb{Z}^{+}$
\item Verify the conjecture in (b) using induction.
\end{enumerate}
\textbf{Base Case:}\\
Let k be 0: $2*0 + 1 = 1^2 => 1 = 1$ so this holds True.\\
We assume p(k)
$1 + 3 + 5 + 7 + 9 + ... + 2k + 1 = (k + 1)^2$.\\
We see that $1 + 3 + 5 + 7 + 9 + ...$ can be replaced with p(k)\\
Let p(k + 1) : $(k + 1)^2 + 2(k + 1) + 1$\\
 = $k^2 + 2k + 1 + 2k + 3$\\
 = $k^2 + 4k + 4$\\
 = $(k + 2)^2$\\
 = $((k + 1) + 1)^2$\\
 So this holds true for our inductions step. Which means that our conjecture is true. \includegraphics[scale=0.70]{billeder/xzibit}


%----------------------------------------------------------------------------------------
% Problem 7
%----------------------------------------------------------------------------------------
\section{Using induction, prove that}
\begin{enumerate}[(a)]
\item $\forall n \in \mathbb{N}, if n\geq 2, then \ n^3-n$ is always divisable by 3
\item $\forall n \in \mathbb{N},n < 2^n$
\end{enumerate}
\textbf{(a)}\\
Base Case:\\
Let n = 2. $2^3 - 2 = 6$ which is divisable by 3.\\

Assume p(k): $k^3-k = 3m$ where m is an arbitrary integer.\\
Let p(k+1):\\
 = $(k + 1)^3 - (k + 1)$\\
 = $k^3 + 3k^2 + 3k + 1 - k - 1$\\
 = $k^3 - k +3k^2 + 3k$\\
 = $3m +3k^2 + 3k$ because $k^3 - k = 3m$ as established by our p(k)\\
 = $3(m +k^2 + k)$\\
 Which is always divisable by 3. So our induction step holds true. \includegraphics[scale=0.70]{billeder/xzibit}\\
\textbf{(b)}\\
Base Case:\\
Let n = 1. $1<2^1$ which is true.\\

Assume p(k): $k < 2^k$\\
Let p(k+1):\\
 = $(k + 1) < 2^{k + 1}$\\
 = $(k + 1) < 2*2^{k}$\\
 = $(k + 1) < 2^{k} + 2^{k}$\\
 = $1 <  + 2^{k} - k$
 = $1 - 2^{k} < 2^{k} - k$.\\
 = $1 - k < 2^{k} - k$ because $k < 2^k$\\
 = $1 < 2^{k}$
  Which holds true for all natural values of k.\includegraphics[scale=0.70]{billeder/xzibit}

