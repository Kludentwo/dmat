\textbf{\Huge Problems}
%----------------------------------------------------------------------------------------
% Problem 1
%----------------------------------------------------------------------------------------
\section{Consider the following statements:}
$(1+2)^2-1^2=2^3$\\
$(1+2+3)^2-(1+2)^2=3^3$\\
$(1+2+3+4)^2-(1+2+3)^2=4^3$\\
\begin{enumerate}[(a)]
\item Based on the three statements given above, what is the next statement suggested by these?
\item What conjecture is suggested by these statements?
\item Verify the conjecture in (b).
\end{enumerate}
\textbf{\Large(a)}\\
We see that the next statement would be: $(1+2+3+4+5)^2-(1+2+3+4)^2=5^3$.
\textbf{\Large(b)}\\
The conjecture is $(1+2+3+4+5+...+n)^2-(1+2+3+4+...+n-1)^2=n^3$.
\textbf{\Large(c)}\\
$(1+2+3+4+5+...+n)^2-(1+2+3+4+...+n-1)^2=n^3$.

%----------------------------------------------------------------------------------------
% Problem 2
%----------------------------------------------------------------------------------------
\section{By an ordered partition of an integer $n\geq 2$ is meant a sequence of positive integers whose sum is n.}
For example, the ordered partitions of 3 are 3, 1 + 2, 2 + 1, 1 + 1 + 1.
\begin{enumerate}[(a)]
\item Determine the ordered partitions of 4.
\item Determine the ordered partitions of 5.
\item Make a conjecture concerning the number of ordered partitions of an integer $n\geq 2$
\end{enumerate}

%----------------------------------------------------------------------------------------
% Problem 3
%----------------------------------------------------------------------------------------
\section{Express the following quantified statement in symbols:}
For every odd integer n, the integer 3n+1 is even.\\

Part (b) \\
Prove that the statement is true.

%----------------------------------------------------------------------------------------
% Problem 4
%----------------------------------------------------------------------------------------
\section{Express the following quantified statement in symbols:}
There exists a positive integer n such that $3n+2^{n-2}$is odd.\\

Part (b) \\
Prove that the statement is true.

%----------------------------------------------------------------------------------------
% Problem 5
%----------------------------------------------------------------------------------------
\section{Prove or disprove: The sum of every five consecutive integers is divisable by 5 and the sum of no six consecutive integers is divisable by 6}

%----------------------------------------------------------------------------------------
% Problem 6
%----------------------------------------------------------------------------------------
\section{Consider the following statements:}
1 = 1,\\
1 + 3 = 4,\\
1 + 3 + 5 = 9,\\
1 + 3 + 5 + 7 = 16,\\
1 + 3 + 5 + 7 + 9 = 25.\\
\begin{enumerate}[(a)]
\item Based on the three statements given above, what is the next statement suggested by these?
\item What conjecture is suggested by these statements?
\item Verify the conjecture in (b) using induction.
\end{enumerate}

%----------------------------------------------------------------------------------------
% Problem 7
%----------------------------------------------------------------------------------------
\section{Using induction, prove that}
\begin{enumerate}[(a)]
\item $\forall n \in \mathbb{N}, if n\geq 2, then n^3-n is always divisable by 3$
\item $\forall n \in \mathbb{N},n < 2^n$
\end{enumerate}
{\Huge\Bat}



