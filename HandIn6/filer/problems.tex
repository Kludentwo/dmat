\textbf{\Huge Problems}
%----------------------------------------------------------------------------------------
% Problem 1
%----------------------------------------------------------------------------------------
\section{Let A = \{a, b, c, d\}. Give an example (with justification) of a relation R on A that has none of the following properties: reflexive, symmetric, transitive.}
\includegraphics[scale=0.70]{billeder/xzibit}

%----------------------------------------------------------------------------------------
% Problem 2
%----------------------------------------------------------------------------------------
\section{A relation R is defined on $\mathbb{Z}$ by aRb if $|a-b|\leq 2$. Which of the properties reflexive, symmetric and transitive does the relation R possess? Justify your answers.}
hest.jpg


%----------------------------------------------------------------------------------------
% Problem 3
%----------------------------------------------------------------------------------------
\section{Let A and B be sets with $|A|=|B|=4$.}
\begin{enumerate}[a.]
\item Prove or disprove: If R is a relation from A to B where $|R|=9$ and $R = R^{-1}$, then $|A=B|$.
\item Show that by making a small change in the statement in (a), a different response to the resulting statement can be optained.
\end{enumerate}
hest.png


%----------------------------------------------------------------------------------------
% Problem 4
%----------------------------------------------------------------------------------------
\section{Let R be a relation defined on $\mathbb{Z}$ by aRb if $a^3 = b^3$. Show that R is an equivalence relation on $\mathbb{Z}$ and determine the distinct equivalence classes.}
hest.gif


%----------------------------------------------------------------------------------------
% Problem 5
%----------------------------------------------------------------------------------------
\section{Let $H = \{2m : m\in \mathbb{Z}\}$. A relation R is defined on the set $\mathbb{Q}^{+}$ of positive rationals by aRb if $a/b\in H$.}
\begin{enumerate}[a.]
\item Show that R is an equivalence relation.
\item Describe the elements in the equivalence class [3].
\end{enumerate}
hest.svg\\
H er basically alle lige numre.

%----------------------------------------------------------------------------------------
% Problem 6
%----------------------------------------------------------------------------------------
\section{Two parts:}
\begin{enumerate}[a.]
\item Prove that the intersection if two equivalence relations on a nonempty set is an equivalence relation.
\item Consider the equivalence relations $R_2$ and $R_3$ defined on $\mathbb{Z}$ by a$R_2$b if $a\equiv b(mod 2)$ and a$R_3$b if $a\equiv b(mod 3)$. By (a), $R_1=R_2\cap R_3$ is an equivalence relation on $\mathbb{Z}$. Determine the destinct equivalence classes in $R_1$.
\end{enumerate}
hest.pdf

%----------------------------------------------------------------------------------------
% Problem 7
%----------------------------------------------------------------------------------------
\section{Let A = \{1,2,3\} and B = \{a,b,c,d\}. Give an example of a relation R from A to B containing exactly three elements such that R is not a function from A to B. Explain why R is not a function.}
hest.m

%----------------------------------------------------------------------------------------
% Problem 8
%----------------------------------------------------------------------------------------
\section{For a function f : A $\leftarrow$ B and subsets C and D of A and E and F of B, prove the following:}
\begin{enumerate}[a.]
\item $f(C \cup D) = f(C) \cup f(D)$
\item $f(C \cap D) \subseteq f(C) \cap f(D)$
\item $f(C) - f(D) \subseteq f(C-D)$
\item $f^{-1}(E \cup F) = f^{-1}(E) \cup f^{-1}(F)$
\item $f^{-1}(E \cap F) = f^{-1}(E) \cap f^{-1}(F)$
\item $f^{-1}(E - F) = f^{-1}(C) - f^{-1}(D)$
\end{enumerate}
hest.m

%----------------------------------------------------------------------------------------
% Problem 9
%----------------------------------------------------------------------------------------
\section{Prove that the function f : R $\leftarrow$ R defined by f(x) = 7x - 2 is bijective.}
hest.m

%----------------------------------------------------------------------------------------
% Problem 10
%----------------------------------------------------------------------------------------
\section{Prove or disprove the following:}
\begin{enumerate}[a.]
\item If two functions f : A $\leftarrow$ B and g : B $\leftarrow$ C are both bijective, then g $\circ$ f : A $\leftarrow$ C is bijective.
\item Let f : A $\leftarrow$ B and g : B $\leftarrow$ C be two functions. If g is onto, then g $\circ$ f : A $\leftarrow$ C is onto.
\item Let f : A $\leftarrow$ B and g : B $\leftarrow$ C be two fuctions. If g is one-to-one, then g $\circ$ f : A $\leftarrow$ C is one-to-one.
\item There exist functions f : A $\leftarrow$ B and g : B $\leftarrow$ C such that f is no onto and g $\circ$ f : A $\leftarrow$ C is onto.
\item There exist functions f : A $\leftarrow$ B and g : B $\leftarrow$ C such that f is no one-to-on and g $\circ$ f : A $\leftarrow$ C is one-to-one.
\end{enumerate}