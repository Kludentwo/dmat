\textbf{\Huge Problems}
%----------------------------------------------------------------------------------------
% Problem 1
%----------------------------------------------------------------------------------------
\section{Let A = \{a, b, c, d\}. Give an example (with justification) of a relation R on A that has none of the following properties: reflexive, symmetric, transitive.}
R = {(a,a),(a,b),(b,c)}.\\
It is not reflexive as it does not contain: (b,b),(c,c),(d,d).\\
It is not symmetric because it does not contain: (b,a), (c,b).\\
It is not transitive because it does not contain: (a,c) although it contains the valid steps to get to (a,c).
\newpage

%----------------------------------------------------------------------------------------
% Problem 2
%----------------------------------------------------------------------------------------
\section{A relation R is defined on $\mathbb{Z}$ by aRb if $|a-b|\leq 2$. Which of the properties reflexive, symmetric and transitive does the relation R possess? Justify your answers.}
We start by checking if it is reflexive:\\
\begin{equation}
aRa = |a-a|\leq 2 = |0|\leq 2 = True
\end{equation}
Next we check if it is symmetric:
\begin{equation}
aRb = |a-b|\leq 2
\end{equation}
\begin{equation}
bRa = |b-a|\leq 2
\end{equation}
Since we know that $|a-b| \equiv |b-a|$, we know that it is symmetric.\\
Lastly we check if it is transitive:
\begin{equation}
aRb \wedge bRc => aRc
\end{equation}
We provide a counterexample:\\
Let a = 0, b = 1, c = 3.
$aRb = |0-1|=1 \leq 2$ and $bRc = |1-3|=2 \leq 2$ implies $aRc = |0-3|= 3 \leq 2$ which is false.\\
We can conclude and say that the relation aRb is reflexive and symmetric but not transitive. 
\newpage
%----------------------------------------------------------------------------------------
% Problem 3
%----------------------------------------------------------------------------------------
\section{Let A and B be sets with $|A|=|B|=4$.}
\begin{enumerate}[a.]
\item Prove or disprove: If R is a relation from A to B where $|R|=9$ and $R = R^{-1}$, then $A=B$.
\item Show that by making a small change in the statement in (a), a different response to the resulting statement can be obtained.
\end{enumerate}
\textbf{a.}:\\
We will disporve the statement.\\
Let A = \{1, 3, 5, 7 \} and B = \{1, 3, 5, 9 \}.  $|A| = 4 = |B|$.\\
Let R = \{(1,1),(1,3),(1,5),(3,1),(3,3),(3,5),(5,1),(5,3),(5,5)\}. $|R|=9$ and $R^{-1} = R$.\\
We see that A = B evaluates to be false as the fourth element in each is not equal.\\
\textbf{b.}:\\
If we let |R|= 16 $(|A| * |B|)$ and R = $R^{-1}$ it would mean that A and B would have to be equal.
\newpage
%----------------------------------------------------------------------------------------
% Problem 4
%----------------------------------------------------------------------------------------
\section{Let R be a relation defined on $\mathbb{Z}$ by aRb if $a^3 = b^3$. Show that R is an equivalence relation on $\mathbb{Z}$ and determine the distinct equivalence classes.}
We see if the relation is and equivalence relation by checking the individual relations:\\
\textbf{Reflexive:}\\
$aRa: a^3=a^3$, which is true\\ 
Hereby the relation is reflexive.\\
\textbf{Symmetrics:}\\
If$aRb: a^3=b^3$\\
then $bRa: b^3=b^3$\\
and hereby the relation is symmetric.\\
\textbf{Transitive:}\\
If $aRb: a^3=b^3$\\
and $bRc: b^3=c^3$\\
then $aRc: a^3=c^3$\\
and hereby the relation is transitive.\\
\\
The distinct equivalence classes are defined as classes or sets of number which are equal/equivalent. We know $a^3=b^3$ only holds if $a=b$. This means that there exists an equivalence class $[a]=\{a\},  \forall a \in \mathbb{Z}$

\newpage
%----------------------------------------------------------------------------------------
% Problem 5
%----------------------------------------------------------------------------------------
\section{Let $H = \{2^m : m\in \mathbb{Z}\}$. A relation R is defined on the set $\mathbb{Q}^{+}$ of positive rationals by aRb if $a/b\in H$.}
\begin{enumerate}[a.]
\item Show that R is an equivalence relation.
\item Describe the elements in the equivalence class [3].
\end{enumerate}
To show that R is an equivalence relation we will show that the relation is reflexive, symmetric and transitive.\\
\textbf{a:}\\
\textbf{Reflexive:}\\
$aRa: \frac{a}{a}=1$ which is in H thereby the relation is reflexive.\\
\textbf{Symmetric:}\\
$aRb: \frac{a}{b}$ where the relation is in H: $\frac{a}{b}=2^m$\\
$bRa: \frac{b}{a}$ here the relations is also in H: $\left(\frac{a}{b}\right)^{-1}=\left(2^m\right)^{-1}$ which is equal to $\frac{b}{a}=2^{-m}$\\
Hereby the relation is symmetric.\\
\textbf{Transitive:}\\
$aRb:\frac{a}{b}$ let this be in H by some m by $2^m$ which is in H\\
$bRc:\frac{b}{c}$ let this also be in H by some $m_2$ by $2^{m_2}$ which is in H\\
$aRc:\frac{a}{c}$ this can be shown as: $\frac{a}{b}*\frac{b}{c}=2^m*2^{m_2}$ which is equal to $\frac{a}{c}=2^{m+m_2}$\\
Therefore the relation is also transitive.\\
\textbf{b:}\\
So the equivalence classe [3] is $\{a\vee b=3:\frac{a}{b}=2^m\}$. Since the relation is an equivalence relation i can choose whichever(a or b) i want so: $b=3,\frac{a}{3}=2^m$ and here we solve for a: $2^m*3$ so the equivalence class $[3]=\{2^m*3:m\in\mathbb{Z}\}$.
\newpage
%----------------------------------------------------------------------------------------
% Problem 6
%----------------------------------------------------------------------------------------
\section{Two parts:}
\begin{enumerate}[a.]
\item Prove that the intersection if two equivalence relations on a nonempty set is an equivalence relation.
\item Consider the equivalence relations $R_2$ and $R_3$ defined on $\mathbb{Z}$ by a$R_2$b if $a\equiv b(mod 2)$ and a$R_3$b if $a\equiv b(mod 3)$. By (a), $R_1=R_2\cap R_3$ is an equivalence relation on $\mathbb{Z}$. Determine the destinct equivalence classes in $R_1$.
\end{enumerate}
\textbf{a:}\\
Let A be a nonempty set.\\
Then let two equivalent relations be:\\
$R_1\subseteq $A$\times$A and $R_2\subseteq $A$\times$A\\
Here we know $R_3=R_1 \cap R_2$\\
We check for the reflexive property:\\
Then let a$\in$A where (a,a)$\in R_1 \cap$ (a,a)$\in R_2$ then (a,a)$\in R_3$ therefore $R_3$ must be reflexive.\\
Next we check for the symmetric property:\\
Let a,b$\in$A where (a,b),(b,a)$\in R_1 \cap$ (a,b),(b,a)$\in R_2$ then (a,b),(b,a)$\in R_3$ therefore $R_3$ must be symmetric.\\
Last we check for the transitive property:\\
Let a,b,c$\in$A where (a,b),(b,c),(a,c) $\in R_1 \cap$ (a,b),(b,c),(a,c)$\in R_2$ then (a,b),(b,c),(a,c) $\in R_3$ therefore $R_3$ must be transitive.\\
\textbf{b:}\\
We know the relation $R_2$ can be written as $2|(a-b)$ and the relation $R_3$ can be written as $3|(a-b)$. The relation $R_2$ is all even numbers which can be written as $2x$ where $x=(a-b)$. The relation $R_3$ is for all a and b of the form (a-b) is divisible by 3 which is equivalent to $3x$ where x is (a-b). The relation of the intersection of $R_2$ and $R_3$ must then be $3(2x)=6x\equiv a=b(mod(6))$ where $x=(a-b)$. So the intersection is all numbers which are divisible by 6.
When we have mod(6) we can have 6 possible outcomes of remainder, namely: 0,1,2,3,4,5. This can be written as 6 possible equivalence classes which will always give us the same outcome:
\begin{center}
$[0]=\{\forall x\in \mathbb{Z}:6x\}$ - This will always give 0 in remainder.\\
$[1]=\{\forall x\in \mathbb{Z}:6x+1\}$ - This will always give 1 in remainder.\\
$[2]=\{\forall x\in \mathbb{Z}:6x+2\}$ - This will always give 2 in remainder.\\
$[3]=\{\forall x\in \mathbb{Z}:6x+3\}$ - This will always give 3 in remainder.\\
$[4]=\{\forall x\in \mathbb{Z}:6x+4\}$ - This will always give 4 in remainder.\\
$[5]=\{\forall x\in \mathbb{Z}:6x+5\}$ - This will always give 5 in remainder.\\
\end{center}

\newpage
%----------------------------------------------------------------------------------------
% Problem 7
%----------------------------------------------------------------------------------------
\section{Let A = \{1,2,3\} and B = \{a,b,c,d\}. Give an example of a relation R from A to B containing exactly three elements such that R is not a function from A to B. Explain why R is not a function.}
For a relation to be a function it must only have one output per input (it must not split).\\
Therefore a relation which is not a function could be:\\
$R={(1,a),(1,b),(1,c)}$\\
\newpage
%----------------------------------------------------------------------------------------
% Problem 8
%----------------------------------------------------------------------------------------
\section{For a function f : A $\rightarrow$ B and subsets C and D of A and E and F of B, prove the following:}
\begin{enumerate}[a.]
\item $f(C \cup D) = f(C) \cup f(D)$
\item $f(C \cap D) \subseteq f(C) \cap f(D)$
\item $f(C) - f(D) \subseteq f(C-D)$
\item $f^{-1}(E \cup F) = f^{-1}(E) \cup f^{-1}(F)$
\item $f^{-1}(E \cap F) = f^{-1}(E) \cap f^{-1}(F)$
\item $f^{-1}(E - F) = f^{-1}(C) - f^{-1}(D)$
\end{enumerate}
\begin{enumerate}[a.]
\item 
\end{enumerate}
\newpage
%----------------------------------------------------------------------------------------
% Problem 9
%----------------------------------------------------------------------------------------
\section{Prove that the function f : R $\rightarrow$ R defined by f(x) = 7x - 2 is bijective.}
For a function to be bijective it needs to be both "onto" and "one-to-one".\\
First we will show that the function is "one-to-one":\\
We will check that only one unique input gives one unique output. Assume two arbitrary number a and b. Then $f(a)=f(b)$ we insert a and b to the functions\\
$7a-2=7b-2$ add 2 to each side and divide by 7\\
$a=b$ Then a must be equal to b and we have that the function must be one to one.
We will see if it is "onto":\\
For a function to be "onto" we need to prove that for every interger y there is and integer x such that f(x)=y. Assume $x=\frac{y+2}{7}$ then $7(\frac{y+2}{7})-2=y$.\\
Therefore the function must be bijective.
\newpage
%----------------------------------------------------------------------------------------
% Problem 10
%----------------------------------------------------------------------------------------
\section{Prove or disprove the following:}
\begin{enumerate}[a.]
\item If two functions f : A $\rightarrow$ B and g : B $\rightarrow$ C are both bijective, then g $\circ$ f : A $\rightarrow$ C is bijective.
\item Let f : A $\rightarrow$ B and g : B $\rightarrow$ C be two functions. If g is onto, then g $\circ$ f : A $\rightarrow$ C is onto.
\item Let f : A $\rightarrow$ B and g : B $\rightarrow$ C be two functions. If g is one-to-one, then g $\circ$ f : A $\rightarrow$ C is one-to-one.
\item There exist functions f : A $\rightarrow$ B and g : B $\rightarrow$ C such that f is no onto and g $\circ$ f : A $\rightarrow$ C is onto.
\item There exist functions f : A $\rightarrow$ B and g : B $\rightarrow$ C such that f is no one-to-on and g $\circ$ f : A $\rightarrow$ C is one-to-one.
\end{enumerate}
\textbf{a.}\\
We start by checking for injection(one-to-one):\\
g $\circ$ f(x) = g $\circ$ f(y)\\
g(f(x)) = g(f(y))\\
Since g is injective(bijective).\\
f(x) = f(y)\\
and since f is injective(bijective).\\
x = y\\
Next up we check for surjection(onto):\\
Since we know that both g and f are surjective(onto):\\
Let $c \in C$. Since g is onto, there is an element $b \in B$ so that g(b) = c. Since f is onto, there is an element $a \in A$ so that f(a) = b. then:\\
\begin{equation}
g \circ f(a) = g(f(a)) = g(b) = c
\end{equation}
Since we started with an arbitrary element $c \in C$ and we found an element $a \in C$ which maps onto it, we have shown $g \circ f$ is surjective. \includegraphics[scale=0.70]{billeder/xzibit}\\
\textbf{b.}\\
f er ikke noedvendigvis onto saa derfor er den false(skriv det fancy)


