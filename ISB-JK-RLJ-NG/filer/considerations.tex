\chapter{System considerations}
The following chapter aims to explain some of the considerations we made when we did the preliminary study. 
\section{Platform}



\section{Technology}
\subsection{Ultrasound}
The first method we discussed was ultrasound. But if we were to choose ultrasound we wouldnt have much data processing since we would just have a transmitter/reciever circuit which has an electrical interface. And wo wouldnt learn much about implementing digital signal processing. Peter also 

\subsection{Laser}
We didnt realy consider laser since we would rather try to play around with sound and measurements of sound. We also figured it would be a lot more complicated for this type of project.

\subsection{Audio}
We went along with hearable sound by suggestion from Peter. Hearable sound is easier to work with since, yes we can hear it, but it is easier to measure and produce. When we record sound we also get a lot of sample which we can process.

\section{Equipment}
\subsection{I/O}
\fixme{If we remember what this was about}
\subsection{Ultrasonic sensor}
We had an ultrasonic sensor of the brand "Ping)))". This didnt really conflict with the concern about using ultrasound since it had a simple electrical interface, but it was not much about digital signal processing but more about timing so we didn't feel it was a good way to go.
\subsection{Speaker/Mic}
If we were to use a speaker combined with a microphone, the first thing we needed was the actual components. We also had to find out how exactly we were to connect these units to the blackfin. The microphone would have to have input biasing as the input channel on microprocessors usually range from 0V to 3.3V or 5V. The speaker would either have to be active or we would be connected to an amplifier. 
