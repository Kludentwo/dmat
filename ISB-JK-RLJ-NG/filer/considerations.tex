\chapter{Intro}
Hest

\chapter{Considerations}
\begin{enumerate}
\item Why blackfin?
\item Distance measurements
\begin{enumerate}

\item \textbf{Ultrasound}\\
The first method we discussed was ultrasound. But if we were to choose ultrasound we wouldnt have much data processing since we would just have a transmitter/reciever circuit which has an electrical interface. And wo wouldnt learn much about implementing digital signal processing. Peter also 

\item \textbf{Laser}\\
We didnt realy consider laser since we would rather try to play around with sound and measurements of sound. We also figured it would be a lot more complicated for this type of project.

\item \textbf{Hearable sound}\\
We went along with hearable sound by suggestion from Peter. Hearable sound is easier to work with since, yes we can hear it, but it is easier to measure and produce. When we record sound we also get a lot of sample which we can process.

\end{enumerate}

\item Display/output
\item Signal considerations
Before we went any further with the project we investigated which kind of signals which was good to send. We did that by generating a signal, made a delay and the cross-correlated these two signals.
We tried three different signal
\begin{enumerate}
\item Chirp\\
We started with a chirp. We did that because of gut feeling. We thought this would be the best signal to send mostly because it was controlled and not random, and the signal changed characteristic over time.
\item Sinusoid
\item White gaussian noise
\end{enumerate}

\item Clock cycles and resources
Regarding resources on the blackfin we are looking to try to make our own implementation through the project and then compare with the built-in function. We are not trying to do it better but rather get the understanding of what is going on.
We also want to investigate the use of DMA on the blackfin so we do not waste resources on moving data only processing it.

\end{enumerate}
