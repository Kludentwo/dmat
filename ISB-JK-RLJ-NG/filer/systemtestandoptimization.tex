\chapter{systemtestandoptimization}
Vi begynder nu at tage de tests som vi har lavet i den sidste periode og bearbejde dem. Alle overvejelser omkring C-koden og egen implementation af xcorr skal også ind her. Særligt de operationsmålinger som vi (desværre) ikke fik taget billeder af. D'oh Rasmus! Der skal også lineariseringen som Johnny har arbejdet på ind her
\section{Test setup}
\subsection{System clocks}
The amount of clocks used by our system is a pointer for how efficient the system is and how much time it takes for the system to run. To find the system clocks add -DDO$\_$CYCLE$\_$COUNTS to the compiler. Add the following code:\\
\begin{lstlisting}
cycle_t start_count;
cycle_t stop_count;
	
START_CYCLE_COUNT(start_count);
>> your code <<
STOP_CYCLE_COUNT(stop_count, start_count);
\end{lstlisting}
\subsection{Full system test}
The setup consists of 2 speakers, a microphone and a blackfin bf533 kit connect to a computer running VisualDSP++. One of the speakers is connected to the blackfins left audioport0 and one is connected to the right audioport0. The microphone is connected to the blackfins left inputport0. 

\section{Test description}
\subsection{System clocks}
Run the program in the VisualDSP++ debugger and put a breakpoint after the STOP part of the cycle counter. Read the cycles with:
\begin{verbatim}
View -> Debug Windows -> Locals
\end{verbatim}
\subsection{Full system test}
A piece of wood is placed at fixed intervals in front of the system. These intervals are in millimetres: 250, 500, 750, 1000, 1250, 1500, 1750, 2000, 2250, 2500. The output is found in the VisualDSP++ debugger by placing a breakpoint after measuring distance. 

\section{Test results}
\subsection{system clocks}
\begin{table}[H]
\centering
    \begin{tabular}{|l|l|}
    \hline
    Variable           & Clock Cycles (mm) \\ \hline
    start$\_$count        & 1385380           \\ \hline
    stop$\_$count         & 285290476         \\ \hline
    Program total      & 283905096         \\ \hline
    \end{tabular}
\end{table}
\subsection{Full system test}
\begin{table}[H]
\centering
    \begin{tabular}{|l|l|}
    \hline
    Real Distance (mm) & Measured Distance (mm) \\ \hline
    250                & 699                    \\ \hline
    500                & 853                    \\ \hline
    750                & 1104                   \\ \hline
    1000               & 1267                   \\ \hline
    1250               & 1579                   \\ \hline
    1500               & 1859                   \\ \hline
    1750               & 2026                   \\ \hline
    2000               & 2132                   \\ \hline
    2250               & 2434                   \\ \hline
    2500               & 2801                   \\ \hline
    \end{tabular}
\end{table}
\section{Optimization}
\subsection{Optimization of distance calculation}
\fixme{Johnny er god til excel og optimisering}
\section{Optimized test results}
The results found in this chapter are from tests done the same way as previously except with the system optimized.
\subsection{Full system test}
\begin{table}[H]
\centering
    \begin{tabular}{|l|l|}
    \hline
    Real Distance (mm) & Measured Distance (mm) \\ \hline
    250                & 210                   \\ \hline
    500                & 462                    \\ \hline
    750                & 654                   \\ \hline
    1000               & 1075                   \\ \hline
    1250               & 1244                   \\ \hline
    1500               & 1671                   \\ \hline
    1750               & 1756                   \\ \hline
    2000               & 2034                   \\ \hline
    2250               & 2316                   \\ \hline
    2500               & 2528                   \\ \hline
    \end{tabular}
\end{table}