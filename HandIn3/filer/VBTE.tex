\chapter{VBTE}
Nedenfor følger design af software til VBTE. Dette er lavet på baggrund af kravspecifikation og systemarkitektur. Bemærk der i dette design dokument blandt andet ikke er beskrevet mixer, pga osv. da deres eneste funktion er "Start();". Derudover er der en betydelig hardware del knyttet til dette modul, der refereres derfor til detaljeret hardware design for yderligere detaljer om VBTE modulet.
\section{Modulets ansvar}
Som beskrevet i systemarkitektur står VBTE'en for at måle vandniveauet i ballasttankene samt at lukke vand ind eller ud af ballasttankene. Hertil er der også en kommunikation med SM modulet indeholende instruktioner.
\section{Klassediagram}
Nedenfor ses klassediagrammet for VBTE. Bemærk at koden dog er i C men for overblikket er der lavet klassediagram.
\begin{figure}
\centering
\includegraphics[width=1\textwidth]{billeder/ClassVBTE}
\caption{På figuren ses klassediagrammet for VBTE \textbf{Billedet skal opdateres}}
\end{figure}

\section{Globale variabler}
\begin{table}[H]
\begin{tabular}{|l|p{10cm}|}
\hline
\cellcolor[gray]{0.8}\textbf{Variabel} &\cellcolor[gray]{0.8} \textbf{Beskrivelse}\\ \hline
\texttt{BurstLengthVal} & Denne variabel er anvendt til at håndtere antallet af perioder burstet bliver sendt med.\\ \hline
\texttt{WaitBurstVar} & Bliver brugt til nonblocking delay til SendBurst funktionen.\\ \hline
\texttt{BurstTimerVal} & Holder på Timerens værdi når et burst er sendt.\\ \hline
\texttt{DistanceTimerVal} & Holder værdien på timeren når et burst er modtaget. \\ \hline
\texttt{CalcDistFlag} & Bliver sat når et burst er modtaget så en afstand kan blive beregnet.\\ \hline
\texttt{BurstFlag} & Bliver sat når et burst bliver sendt og hevet ned når et burst er modtaget. Dette sker for ikke at få flere detektioner på samme signal.\\ \hline
\end{tabular}
\end{table}
\section{Metode- og klassebeskrivelser}

%Init h-fil%%%%%%%%%%%%%%%%%%%
\subsection{Init}
\subsubsection{Ansvar}
Denne header har til ansvar at initiere alle moduler og blokke på PSoC'en. Disse funktioner hentes fra PSoC'ens API.
\subsubsection{Funktionsbeskrivelser}
\begin{table}[H]
\begin{tabular}{l p{12.5cm}}
\multicolumn{2}{l}{\texttt{\textcolor{blue}{void} Init(  \texttt{\textcolor{blue}{uint8}* WriteBuffer}, 
\texttt{\textcolor{blue}{uint8}* ReadBuffer}, \texttt{\textcolor{blue}{uint8} BufferSize} )}} \\
\hline
Beskrivelse:& Funktionen anvender API'et fra alle PSoC blokke anvendt i designet og kalder deres start funktion. Derudover initierer den også read- og writebuffer til I2C modulet. \\
Parametre:&\texttt{\textcolor{blue}{uint8}* WriteBuffer}\\
&\texttt{\textcolor{blue}{uint8}* ReadBuffer}\\
&\texttt{\textcolor{blue}{uint8} BufferSize} \\
Returværdi:&Ingen\\
\end{tabular}
\end{table}

%Ventil h-fil%%%%%%%%%%%%%%%%
\subsection{Valve}
\subsubsection{Ansvar}
Denne header har til ansvar at styre ventilerne ud fra "state"-variablen modtaget fra I2C\_handle. Headeren benytter PSoC-API'et til kontrol registre..
\subsubsection{Funktionsbeskrivelser}
\begin{table}[H]
\begin{tabular}{l p{12.5cm}}
\multicolumn{2}{l}{\texttt{\textcolor{blue}{void} ChangeState( \textcolor{blue}{uint8} state )}} \\
\hline
Beskrivelse:& Funktionen modtager state som indeholder indformationer om ventilerne skal være lukkede eller hvilken ventil der skal være åben. Den benytter PSoC5 API'et for kontrol register.\\
Parametre:&\texttt{\textcolor{blue}{uint8} state}\\
Returværdi:&Ingen\\
\end{tabular}
\end{table}

%Dist h-fil%%%%%%%%%%%%%%%%
\subsection{Dist}
\subsubsection{Ansvar}
Denne header har til ansvar at sende burst, beregne afstanden samt at omregne afstanden til procent.
\subsubsection{Funktionsbeskrivelser}
\begin{table}[H]
\begin{tabular}{l p{12.5cm}}
\multicolumn{2}{l}{\texttt{\textcolor{blue}{void} SendBurst( \textcolor{blue}{void} )}} \\
\hline
Beskrivelse:&Denne metode aktiverer en 40kHz clock og tæller perioderne op til 10, lukker for burstet og ligger timerens værdi ind i den globale variabel BurstTimerVal. Herefter sættes BurstFlag'et.\\
Parametre:&Ingen\\
Returværdi:&Ingen\\
\end{tabular}
\end{table}

\begin{table}[H]
\begin{tabular}{l p{12.5cm}}
\multicolumn{2}{l}{\texttt{\textcolor{blue}{float} CalculateDistance( \textcolor{blue}{void} )}} \\
\hline
Beskrivelse:&Denne metode anvender BurstTimerVal og DistanceTimerVal til at finde ud af hvor mange clocks der er gået fra burstet er blevet sendt til det igen er blevet registreret.\\
Parametre:&Ingen\\
Returværdi:&\texttt{\textcolor{blue}{float} DistanceMM}\\
\end{tabular}
\end{table}

\begin{table}[H]
\begin{tabular}{l p{12.5cm}}
\multicolumn{2}{l}{\texttt{\textcolor{blue}{uint8} ConvertMMtoPercent( \textcolor{blue}{float} )}} \\
\hline
Beskrivelse:&Metoden modtager afstanden i millimeter og returnerer hvor mange \% der er i tanken\\
Parametre:&\texttt{\textcolor{blue}{float} DistanceMM}\\
Returværdi:&\texttt{\textcolor{blue}{uint8} DistancePercent}\\
\end{tabular}
\end{table}

%I2C h-fil%%%%%%%%%%%%%%%%
\subsection{I2Chandle}
\subsubsection{Ansvar}
Denne header har til ansvar at håndtere I2C. Den kigger om der er kommet noget i writebufferen. Er der kommet noget i bufferen kigger den efter hvilket tilstand det er der skal ændres til og så smider den afstanden i procent i read bufferen.
\subsubsection{Funktionsbeskrivelser}
\begin{table}[H]
\begin{tabular}{l p{12.5cm} }
\multicolumn{2}{l}{\texttt{\textcolor{blue}{uint8} I2C\_handle(  \texttt{\textcolor{blue}{uint8}* WriteBuffer}, 
\texttt{\textcolor{blue}{uint8}* ReadBuffer}, \texttt{\textcolor{blue}{uint8} BufferSize} )}} \\
\hline
Beskrivelse:& Funktionen anvender API'et fra I2C blokken i PSoC miljøet. Med disse tjekker den om der er fyldt nyt i bufferen og aflæse dette. Herfer kalder den funktionen I2C\_decode til at afkode beskeden fra SM. Herefter klargøres readbufferen til at sende vandniveau tilbage. \\
Parametre:&\texttt{\textcolor{blue}{uint8}* WriteBuffer}\\
&\texttt{\textcolor{blue}{uint8}* ReadBuffer}\\
&\texttt{\textcolor{blue}{uint8} BufferSize} \\
Returværdi:&\texttt{\textcolor{blue}{uint8} State}\\
\end{tabular}
\end{table}

%ADC h-fil%%%%%%%%%%%%%%%%
\subsection{PSoC-API::ADC}
\subsubsection{Ansvar}
Denne header er kun beskrevet fordi der er implementeret en funktionalitet i dette API. Der er under ADC
\subsubsection{Beskrivelse}
Inde i ADC'ens interrupt header tilføjes funktionalitet så der, hver gang der bliver samplet, bliver valideret på om der er en detektion. Er der en detektion sættes flaget til udregning af afstand samt flaget for burst sættes til 0 igen. For at gøre dette anvendes funktionerne fra API'et for ADC'en.

\subsection{State Machine}
Nedenfor ses statemachine der beskriver det overordnede flow i VBTE programmet.
\begin{figure}[H]
\centering
\includegraphics[width=1\textwidth]{billeder/STMVBTE}
\caption{Statemachine for VBTE program}
\end{figure}

\subsection{Timing Diagram}
Nedenfor ses timing diagram for en ultralydspuls til afstandsmåling
\begin{figure}[H]
\centering
\includegraphics[scale=1]{billeder/TimingdiagramVBTE}
\caption{Timing diagram for VBTE ultralydspuls}
\end{figure}

\subsection{Interrupt rutiner}
I dette afsnit beskrives interrupt rutinerne i VBTE programmet. 
\subsubsection{isr\_dist}
Isr\_dist har til ansvar at tælle den globale variabel \texttt{WaitBurstVar} op. Den bliver triggeret af en clock på 1kHz. Hver gang variablen havner over 500 bliver \texttt{SendBurst();} funktionen kaldt og variablen bliver nulstillet.
\subsubsection{isr\_counter}
Isr\_counter tæller variablen \texttt{BurstLengthVal} op. Denne anvendes til at styre at der kun bliver sendt 10 peroder i et burst.