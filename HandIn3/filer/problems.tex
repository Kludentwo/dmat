\chapter{Problems}
%----------------------------------------------------------------------------------------
% Problem 1
%----------------------------------------------------------------------------------------
\section{Disprove the statement: $n\in{0,1,2,3,4}, then 2^n + 3^n+n(n-1)(n-2)$ is prime.}
We assume $2^n + 3^n+n(n-1)(n-2)$ produces a prime for n = 4.\\
Therefore:
\begin{equation}
2^4+3^4+4(4-1)(4-2) = 16 + 81 + 24 = 121 = 11^2
\end{equation}
$11^2$ is not a prime.
This disproves the original statement with a counterexample.

%----------------------------------------------------------------------------------------
% Problem 2
%----------------------------------------------------------------------------------------
\section{Let $a,b\in\mathbb{Z}$ Disprove the statement: if ab and $(a+b)^2$ are of opposite parity, then $a^2b^2$ and a+ab+b are of opposite parity.}
We will prove this statement with a counterexample:
Let $a=2k+1$ and $b=2k+1$
Then
\begin{equation}
(2k+1)*(2k+1) = 4k^2+4k+1 = 2(2k^2+k)+1
\end{equation}
and
\begin{equation}
((2k+1)+(2k+1))^2 = (4k+2)^2 = 16k^2+16k+4 = 2(8k^2+8k+2)
\end{equation}
are of opposite parity, then
\begin{equation}
(2k+1)^2*(2k+1)^2 = (4k^2+4k+1)*(4k^2+4k+1) = 16k^4 + 32k^3 + 24k^2 + 8k + 1 = 2(8k^4+16k^3+12k^2+4k)+1
\end{equation}
and
\begin{equation}
(2k+1)+(2k+1)*(2k+1)+(2k+1) = (2k+1)+4k^2+4k+1+(2k+1) =  4k^2+8k+3 = 2(2k^2+4k+1)+1
\end{equation}
are of opposite parity. But we clearly see that both $a^2*b^2$ and $a+ab+b$ are same parity - odd.

%----------------------------------------------------------------------------------------
% Problem 3
%----------------------------------------------------------------------------------------
\section{Let $a,b\in\mathbb{R}^+$. Use a proof by contradiction to prove that x<y, then $\sqrt{x}<\sqrt{y}$}
So to prove by contradiction we need to negate one of the statements, so: $x<y, then \sqrt{x}\geq \sqrt{y}$\\
Let $x=a^2 \wedge y=(b^2)$ and substitute this into the contadicted statement: $(a^2)<(b^2)$ then $a\geq b$ and that can never be true. Therefore the original statement must be true.

%----------------------------------------------------------------------------------------
% Problem 4
%----------------------------------------------------------------------------------------
\section{Prove that there is no largest negative rational number.}
(Note: -1 is larger than -2.)

%----------------------------------------------------------------------------------------
% Problem 5
%----------------------------------------------------------------------------------------
\section{Prove that there exists no pisitive integer x such that 2x < $x^2$ < 3x.}
hest.jpg

%----------------------------------------------------------------------------------------
% Problem 6
%----------------------------------------------------------------------------------------
\section{Prove that if n is an odd integer, then 7n-5 is even by}
\begin{itemize}
\item[a)] direct proof,
\item[b)] proof by contrapositive,
\item[c)] proof by contradition.
\end{itemize}
hest.png

%----------------------------------------------------------------------------------------
% Problem 7
%----------------------------------------------------------------------------------------
\section{Show that there exist two distinct irrational numbers a and b such that $a^b$ is rational.}
hest.m

%----------------------------------------------------------------------------------------
% Problem 8
%----------------------------------------------------------------------------------------
\section{Disprove the statement: There is an integer n such that $n^4+n^3+n^2+n$ is odd.}
hest.exe