\textbf{\Huge Problems}
%----------------------------------------------------------------------------------------
% Problem 1
%----------------------------------------------------------------------------------------
\section{Disprove the statement: $n\in{0,1,2,3,4}, then 2^n + 3^n+n(n-1)(n-2)$ is prime.}
We assume $2^n + 3^n+n(n-1)(n-2)$ produces a prime for n = 4.\\
Therefore:
\begin{equation}
2^4+3^4+4(4-1)(4-2) = 16 + 81 + 24 = 121 = 11^2
\end{equation}
$11^2$ is not a prime.
This disproves the original statement with a counterexample.

%----------------------------------------------------------------------------------------
% Problem 2
%----------------------------------------------------------------------------------------
\section{Let $a,b\in\mathbb{Z}$ Disprove the statement: if ab and $(a+b)^2$ are of opposite parity, then $a^2b^2$ and a+ab+b are of opposite parity.}
We will prove this statement with a counterexample:
Let $a=2k+1$ and $b=2k+1$
Then
\begin{equation}
(2k+1)*(2k+1) = 4k^2+4k+1 = 2(2k^2+k)+1
\end{equation}
and
\begin{equation}
((2k+1)+(2k+1))^2 = (4k+2)^2 = 16k^2+16k+4 = 2(8k^2+8k+2)
\end{equation}
are of opposite parity, then
\begin{equation}
(2k+1)^2*(2k+1)^2 = (4k^2+4k+1)*(4k^2+4k+1) = 16k^4 + 32k^3 + 24k^2 + 8k + 1 = 2(8k^4+16k^3+12k^2+4k)+1
\end{equation}
and
\begin{equation}
(2k+1)+(2k+1)*(2k+1)+(2k+1) = (2k+1)+4k^2+4k+1+(2k+1) =  4k^2+8k+3 = 2(2k^2+4k+1)+1
\end{equation}
are of opposite parity. But we clearly see that both $a^2*b^2$ and $a+ab+b$ are same parity - odd.

%----------------------------------------------------------------------------------------
% Problem 3
%----------------------------------------------------------------------------------------
\section{Let $a,b\in\mathbb{R}^+$. Use a proof by contradiction to prove that x<y, then $\sqrt{x}<\sqrt{y}$}
So to prove by contradiction we need to negate one of the statements, so: $x<y, then \sqrt{x}\geq \sqrt{y}$\\
Let $x=a^2 \wedge y=b^2$ and substitute this into the contadicted statement: $(a^2)<(b^2)$ then $a\geq b$ and that can never be true. Therefore the original statement must be true.

%----------------------------------------------------------------------------------------
% Problem 4
%----------------------------------------------------------------------------------------
\section{Prove that there is no largest negative rational number.}
(Note: -1 is larger than -2.)\\
We try to prove this by contradiction. So we say there IS a largest negative rational number let that be n.
Then consider $k=n-n/2$ then $k>n$. So in conclusion we see there cannot be a largest negative rational number.

%----------------------------------------------------------------------------------------
% Problem 5
%----------------------------------------------------------------------------------------
\section{Prove that there exists no positive integer x such that 2x < $x^2$ < 3x.}
The statement can be reduced to: $2x<x*x<3x\rightarrow2<x<3$\\
And since x must not be 3 nor 2 and it must be an integer there exists no positive interger for which $2x<x^2<3x$.

%----------------------------------------------------------------------------------------
% Problem 6
%----------------------------------------------------------------------------------------
\section{Prove that if n is an odd integer, then 7n-5 is even by}
\begin{itemize}
\item[a)] Direct proof,
We know an odd integer is of the form $n=2a+1$.\\
We substitute: $7(2a+1)-5\rightarrow14a+7-5\rightarrow2(7a+1)$\\
And the result is of the even form $2a$ therefore the statement must be true.
\item[b)] Proof by contrapositive,
To make a contrapositive proof we change the original statement to its equivelent statement: $P \Rightarrow Q \equiv \neg Q \Rightarrow \neg P$\\
The contrapositive statement is: $if 7n-5 is odd then n i even$\\

\item[c)] proof by contradition.
We assume that $\exists x \in S:$ where S is the set of all odd integers and 7x-5 is odd.\\
Let x be 1.
\begin{equation}
7*1-5 = 2
\end{equation}
2 is not an odd number.
This disproves our assumptions.
Which proves our original statement that if n is an odd integer, then 7n-5 is even.
\end{itemize}

%----------------------------------------------------------------------------------------
% Problem 7
%----------------------------------------------------------------------------------------
\section{Show that there exist two distinct irrational numbers a and b such that $a^b$ is rational.}
hest.m

%----------------------------------------------------------------------------------------
% Problem 8
%----------------------------------------------------------------------------------------
\section{Disprove the statement: There is an integer n such that $n^4+n^3+n^2+n$ is odd.}
We make 2 cases:\\
$n = 2k$ for even integers\\
$n = 2k + 1$ for odd integers\\
for even:
\begin{equation}
(2k)^4+(2k)^3+(2k)^2+(2k) = 16k^4+8k^3+4k^2+2k = 2(8k^4+4k^3+2k^2+k)
\end{equation}
for odd:
\begin{equation}
(2k+1)^4+(2k+1)^3+(2k+1)^2+(2k+1) = 16k^4+40k^3+40k^2+20k+4 = 2(8k^4+20k^3+20k^2+10k+2)
\end{equation}
Both cases end up in even form. This disproves the statement that:\\
There is an integer n such that $n^4+n^3+n^2+n$ is odd.
 
