\textbf{\Huge Problems}
%----------------------------------------------------------------------------------------
% Problem 1
%----------------------------------------------------------------------------------------
\section{Which of the following sets are well-ordered? (Why/why not?)}
\begin{enumerate}[a.]
\item $S={x\in\mathbb{Q}\ :\ x\geq-10}$
\item $S={-2,-1,0,1,2}$
\item $S={x\in\mathbb{Q}\ :\ -1\leq x\leq 1}$
\item $S={p\ : \ p is prime} = {2,3,5,7,9,11,13,...}$
\end{enumerate}


%----------------------------------------------------------------------------------------
% Problem 2
%----------------------------------------------------------------------------------------
\section{Use mathematical induction to prove that $1+5+9+...+(4n-3)=2n^2-n$ for every positive integer n.}


%----------------------------------------------------------------------------------------
% Problem 3
%----------------------------------------------------------------------------------------
\section{Prove that $2^n>n^3$ for every integer $n\geq 10$}
Note: you will need to really work with inequalities.


%----------------------------------------------------------------------------------------
% Problem 4
%----------------------------------------------------------------------------------------
\section{Use the method for minimum counterexample to prove that $3|(2^{2n}-1)$ for every positive integer n.}


%----------------------------------------------------------------------------------------
% Problem 5
%----------------------------------------------------------------------------------------
\section{Use the Strong Principle of Mathmatical Induction to prove the following:}
Let $S={i\in\mathbb{Z}\ : \ i\geq2}$\\
and let P be a subset of S with the properties that $2,3\inP$ and if $n\inS$, then either $n\inP$ or $n=ab$, where $a,b\inS$.
Then every element of S either belongs to P or it can be expressed as a product of elements of P.\\
Note: read Theorem 11.17, though the proof of 11.17 is not the proof of this question.


%----------------------------------------------------------------------------------------
% Problem 6
%----------------------------------------------------------------------------------------
\section{Use the Strong Principle of Mathematical Induction to prove that for each integer $n\geq 12$, there are non-negative integers a and b such that $n = 3a + 7b$.} 
Note: this uses generalized strong induction and minimum counterexamples.



 
