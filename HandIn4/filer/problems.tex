\textbf{\Huge Problems}
%----------------------------------------------------------------------------------------
% Problem 1
%----------------------------------------------------------------------------------------
\section{Which of the following sets are well-ordered? (Why/why not?)}
\begin{enumerate}[a.]
\item $S={x\in\mathbb{Q}\ :\ x\geq-10}$
\item $S={-2,-1,0,1,2}$
\item $S={x\in\mathbb{Q}\ :\ -1\leq x\leq 1}$
\item $S={p\ : \ p is prime} = {2,3,5,7,9,11,13,...}$
\end{enumerate}
First we look at the definition:\\
A set is well-ordered if every nonempty subset has a least element.\\
Then we look at the sets:\\
\begin{enumerate}[a.]
\item S is not well-ordered set because we can make a subset that doesn't contain a least element. e.g. x > -10. 
\item S is a well-ordered set because we can always produce a least element from the subsets of S.
\item S is not a well-ordered set because we can make a subset that doesn't contain a least element. e.g. 0 < x $\leq$ 1.
\item S is a well-ordered set because we can always produce a least element from the subsets of S.
\end{enumerate}

%----------------------------------------------------------------------------------------
% Problem 2
%----------------------------------------------------------------------------------------
\section{Use mathematical induction to prove that $1+5+9+...+(4n-3)=2n^2-n$ for every positive integer n.}
We have $n\in \mathbb{Z}^{+}$.\\
We start by establishing the base case:
\begin{equation}
(4n-3) = 2n^2-n \Rightarrow (4*1-3) = 2*1^2-1 \Rightarrow 1 = 1
\end{equation}
$p(k)\Rightarrow p(k+1)$
Assume $p(k)$\\
$p(k+1) \equiv p(k)+(4n-3) \equiv 2k^2-k+(4(k+1)-3)$\\
\begin{equation}
2k^2-k+4k+1
\end{equation}
\begin{equation}
2k^2-k+4k+1+1-1
\end{equation}
\begin{equation}
2k^2+4k+2-k-1
\end{equation}
\begin{equation}
2(k^2+2k+1)-(k+1)
\end{equation}
\begin{equation}
2(k^2+2k+1)-(k+1)
\end{equation}
\begin{equation}
2(k+1)^2-(k+1)
\end{equation}
We see that his is our assumption for k + 1


%----------------------------------------------------------------------------------------
% Problem 3
%----------------------------------------------------------------------------------------
\section{Prove that $2^n>n^3$ for every integer $n\geq 10$}
Note: you will need to really work with inequalities.
Assume $m$ such that $2^n\leq n^3$\\
Initial:$2^{10}=1024 > 1000 = 10^3$\\
$m$ must be bigger than 10.\\
$m = k + 1$ where $10\leq k < m$\\
$2^k > k^3$\\
$2^m = 2^{k+1}$\\
$\ \ = 2*2^k$\\
$\ \ > 2*k^3$ because $2^k > k^3$\\
$\ \ = k^3+k^3$\\
$\ \ \geq k^3+10k^2$ because $10\leq k$\\
$\ \ =k^3+3k^2+7k^2$\\
$\ \ >k^3+3k^2+3k+4k^2$ because $3k^2 > 3k$\\
$\ \ >k^3+3k^2+3k+1$ because $4k^2 > 1$\\
$\ \ =(k+1)^3$\\
$\ \ =m^3$\\
Which is a contradiction!\\
This proves that $2^n > n^3$ for every integer $n\geq 10$
%----------------------------------------------------------------------------------------
% Problem 4
%----------------------------------------------------------------------------------------
\section{Use the method for minimum counterexample to prove that $3|(2^{2n}-1)$ for every positive integer n.}
Base case: $p(1) = 2^{2*1}-1 = 3$ so True.\\
Assume: $p(k) = 2^{2k}-1 = 3m$ where m is an integer.\\

$p(k+1) = 2^{2(k+1)}-1$\\
$\ \  = 2^{2k+2}-1$\\
$\ \  = 2*2*2^{2k}-1$\\
$\ \  = 4*2^{2k}-1$\\
$\ \  = 4*2^{2k}-1+4-4$\\
$\ \  = 4*2^{2k}-1+4-4$\\
$\ \  = 4*2^{2k}+3-4$\\
$\ \  = 4*(2^{2k}-1)+3$\\
$\ \  = 4*(2^{2k}-1)+3$\\
$\ \  = 4*(3M)+3$\\
$\ \  = 3*(4M+1)$\\
Which is divisable by 3.\\
so p(n) is always divisable by 3 forall $n\in \mathbb{Z}^{+}$

%----------------------------------------------------------------------------------------
% Problem 5
%----------------------------------------------------------------------------------------
\section{Use the Strong Principle of Mathmatical Induction to prove the following:}
Let $S={i\in \mathbb{Z}\ : \ i\geq 2}$ and let P be a subset of S with the properties that 
$2 , 3 \in P$ and if 
$n\in S$, then either 
$n\in P$ or 
$n=ab$, where 
$a,b\in S$.
Then every element of S either belongs to P or it can be expressed as a product of elements of P.\\
Note: read Theorem 11.17, though the proof of 11.17 is not the proof of this question.


%----------------------------------------------------------------------------------------
% Problem 6
%----------------------------------------------------------------------------------------
\section{Use the Strong Principle of Mathematical Induction to prove that for each integer $n\geq 12$, there are non-negative integers a and b such that $n = 3a + 7b$.} 
Note: this uses generalized strong induction and minimum counterexamples.



 
